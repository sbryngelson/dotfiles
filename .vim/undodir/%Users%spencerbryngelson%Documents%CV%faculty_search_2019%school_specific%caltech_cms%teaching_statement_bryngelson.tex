Vim�UnDo��)2�b�X0��H4�~�ʸ�qDX������V�@�]�$�_��������V]�$����,@\section*{Course strengths and future development \sout{\hfill}}\iftoggle{mech}{tI have taught solid and fluid mechanics and mechanical design courses in the past, and am eager to teach them again.�My broad academic and research background also qualifies me to teach courses on dynamics, control, computational methods, heat transfer, and thermodynamics. }{\iftoggle{bio}{�My broad academic and research background qualifies me to teach courses on biomechanics, biomaterials, engineering design, molecular and  cellular biology, biotransport, computational methods, and biomembranes, among others.}{\iftoggle{chem}{tI have taught solid and fluid mechanics and mechanical design courses in the past, and am eager to teach them again.�My broad academic and research background also qualifies me to teach courses on thermodynamics, heat transfer, transport phenomena, soft matter, and more.}{}}}\iftoggle{mech}{MI also envision developing a new course to complement your existing catalog. SThis course would focus on the \textbf{computational methods for multiphase flows}.9Most practical engineering flows involve multiple phases.KSpecial computational techniques are often required to compute these flows.=However, these methods are often omitted from the curriculum.XI would present an introduction to the tools used for modern multiphase flow simulation.jThis would be a graduate-level course, though I would also accommodate upper-level undergraduate students.}{\iftoggle{bio}{MI also envision developing a new course to complement your existing catalog. IThis course would focus on the \textbf{rheology of biological materials}.dRheology unifies the solid and fluid characteristics of complex materials such as blood and tissues.nDeveloping faithful models for the dynamics of even simple cells or physiology requires a command of rheology.iFurther, rheological models often serve as a basis for reduced-order representations of biological flows.jThis would be a graduate-level course, though I would also accommodate upper-level undergraduate students.}{\iftoggle{chem}{MI also envision developing a new course to complement your existing catalog. PThis course would focus on the \textbf{computational methods for complex flows}.FMost practical engineering flows involve complex fluids and materials.KSpecial computational techniques are often required to compute these flows.=However, these methods are often omitted from the curriculum.UI would present an introduction to the tools used for modern copmlex flow simulation.jThis would be a graduate-level course, though I would also accommodate upper-level undergraduate students.}{}}}5��